\documentclass[10pt,a4paper]{article}
\usepackage[utf8]{inputenc}
\usepackage{amsmath}
\usepackage{amsfonts}
\usepackage{amssymb}
\usepackage{graphicx}


\usepackage[usenames,dvipsnames]{xcolor}
\newcommand{\Wilma}[1]{\textcolor{Magenta}{#1}}
\newcommand{\HW}[1]{\textcolor{Green}{#1}}
\newcommand{\Jo}[1]{\textcolor{Cyan}{#1}}
\newcommand{\Elena}[1]{\textcolor{Orange}{#1}}
\newcommand{\Answer}[1]{\textcolor{Gray}{#1}}

\usepackage{soul}

\begin{document}
\section*{Reviewer's Comments on "Action-based Dynamical Modeling for the Milky Way Disk: The Influence of Spiral Arms" (14. November 2016)"}

\paragraph{Summary of the paper:} The article is a follow up of Paper I, which presented a method (RoadMapping) to constrain the MW potential from data. In Paper I, mock data were obtained, assuming
the potential is axisymmetric, which was a simplistic approximation. In this second
paper, the mock data are obtained from N-body simulations, showing spiral arms.

\paragraph{Main concerns, Aspect 1 \& 2:} (1) However, there is no bar in the simulations, which again is a very simplistic approximation.  (2) Also the spiral structure is faint, and uncontrasted with respect to
the total potential.

\Answer{That our simulation does not have a bar is indeed a limitation of the methodology used in this paper. We made that clear in the paper, and now mention it in the abstract, in the introduction, in a new section 2.5, which compares the simulation and the MW, and we also moved the paragraph on the bar in the discussion to a more prominent position in Section 5.5. Ultimately we are not trying to analyze the most realistic non-axisymmetric MW simulation, but rather one with very strong spiral arms.}

\Answer{The referee's second concern, that the spiral structure is too faint, is however not the case. We will discuss this in more detail below.}
 
\paragraph{Summary of the paper (continued):} The authors have done a thorough and very detailed analysis of the mock data, varying the center of the volume (Sun at the inter-arm, or on an arm, etc..), and
checked how much the volume of data considered changes the results. This is very
useful to prepare the exploitation of Gaia data.

\paragraph{Aspect 2: Amplitude of spiral arms (continued):} The present study concludes that RoadMapping is still relatively valid when considering spiral structure, however this referee is not quite convinced that this
will apply to the MW, since the N-body models considered here are so axisymmetric,
and the spiral structure amplitude so negligible. In the real MW world, the contrast
of spiral arms and bars will be much higher in the potential, and will perturb much
more the orbits.  

\Answer{We do not think that it is the case that the spiral arms in this simulation are much weaker than in the MW. (a) Figure 5 compares the stellar data with an axisymmetric best-fit model and it is very obvious that there are very strong non-axisymmetries and an excess of stars with $v_R\sim 50~\text{km/s}$ due to the spiral arms. Non-axisymmetric motions in the MW are around $10-20~\text{km/s}$. (b) Figure 2 \Wilma{[Check]} demonstrates that the surface density variations with $\phi$ due to the spiral arms in the disk only are very strong, especially around $R=5~\text{kpc}$. (Figure 2 is a new plot we included in the paper to make this very clear.) Spiral arms around this radius affect almost all of our survey volumes, also those that are centered at $R=8~\text{kpc}$ for $r_\text{max}\gtrsim2~\text{kpc}$. (c) Figure 10 shows the total (incl. dark matter) circular velocity curve and radial surface density profile within a small $\Delta \phi$ wedge. The spiral arms are therefore strong perturbations also in the total potential.}

\paragraph{Aspect 3: dark matter fraction:} Indeed, the model considered here is dominated by dark matter (stellar disk mass fraction of 4\%, also stabilized by a spherical bulge of mass 25\% of the stellar disk mass). The amplitude of the spiral arms obtained certainly brings a negligible
perturbation on the total gravitational potential of the galaxy model. In the
realistic MW, the dark matter fraction within the solar radius has been established
to be much less (the MW inside the solar radius is dominated by baryons, e.g. Barros
et al 2016 A\&A 593, A108 and references therein.). The dominance of baryons and the
much higher gas fraction make the disk much more self-gravitating and unstable, and
the bar amplitude is non-negligible, as well as the strong spiral arms associated.

\Answer{The stellar disk fraction of 4\% is only the fraction of the \emph{total} mass of the dark matter halo. Locally the disk dominates over dark halo and bulge, especially between $R=3-7~\text{kpc}$. The rotational support at $2.2R_s$ is $\sim50\%$ as shown in Figure \ref{fig:rot_support} below. (The upper panel of Figure \ref{fig:rot_support}, which shows the decomposition of the rotation curve of the reference potential model \texttt{DEHH-Pot}, was also included in the paper as Figure \Wilma{[Check]} to also make this clear to the reader.) This high disk fraction leads to the formation of four strong arms, which we also demonstrate in the Fourier decomposition in Figure \Wilma{[Check]}, which we included in the paper. To make absolutely clear in what aspects the simulation and the MW are similar and in what not, we have included a new Section 2.5 in the paper, where we discuss the mass budget, the disk fraction, the number of spiral arms and the strength of the arms. We also compare the simulation with the MW models from Barros et al. (2016) and find that one of their two MW models is qualitatively not too different to our simulation. The only major deviations from the MW are, that there is no bar (see above) and no gas component in the simulation, and the disk is thinner than in the MW. The gas is not a problem, as it can easily be included in the modelling of the MW. It's potential perturbation due to gas is not expected to be much stronger than the potential perturbations investigated in this work. That the disk is thinner in this simulation means, that its vertical velocity dispersion is lower and it is more easily affected by perturbations. In the MW RoadMapping will implicitly deal with tracers of different disk components and it is also easy to include, if necessary, a thick disk in the potential model.}

%====================
\begin{figure}[!htbp]
\centering
\includegraphics[width=0.7\columnwidth]{fig/plot_vcirc_decomposed.pdf}
\caption{\Answer{Circular velocity curve and fractional contribution to the radial force (rotational support) of the disk, halo and bulge components of the \texttt{DEHH-Pot}, the symmetrized best fit to the $N$-body simulation. This demonstrates that the simulation is a disk-dominated spiral galaxy.}}
\label{fig:rot_support}
\end{figure}
%====================

\paragraph{Aspect 2: Amplitude of spiral arms (continued):} The model considered here, almost axisymmetric (the contrast of the spiral structure is in its large majority less than or equal to 10\% in stellar surface density, as shown in Figure 2),  is therefore very far from the reality of the MW, and can be highly deceiving for the MW results to come.

\Answer{}

\paragraph{Aspect 4: Conservation of actions} The fact that the DF is still expressed in terms of integrals of motion, which should not be any more conserved, is a questionable issue.

\Answer{Actions in an axisymmetric potential are integrals of motions and according to the Jeans Theorem any steady-state distribution function solution to the Boltzmann equation becomes $DF(\vec{x},\vec{v},t) \longrightarrow DF(\vec{J})$. In an axisymmetric potential superimposed with non-axisymmetric perturbations, actions are indeed not fully conserved anymore. It is however still possible to estimate the local current actions (using an axisymmetric approximation to the potential) and treat $(\vec{J},\vec{\theta})$ simply as phase-space coordinates analogous to $(\vec{x},\vec{v})$. It is interesting to test, if using the $qDF(\vec{J})$ locally as model for the true $DF(\vec{x},\vec{v},t)\approx DF(\vec{J},\vec{\theta},t)$ could still tell us something about the true potential. This is one of the aspects of action-based modelling we wanted to test in this study---and which we found to be indeed the case. To make this motivation more clear in the paper, we have included a paragraph in the introduction and another one in Section 3.2 about the distribution function model.}

\paragraph{Aspect 5: Radial migration and MAP concept:} Moreover, we know that spiral structure will trigger radial migration of stars, the
more so that there velocity dispersion is small. So for a given stellar population,
with a given metal abundance (and alpha/FE ratio), we expect that a dispersion will
occur in orbits, and the MAP concept will no longer be valid. \hl{How to answer to this?}

\paragraph{Question 1: Corotation of spiral arms} About this problem, it will be also useful to mention where is the corotation of the
spiral in the N-body model, is it near the Sun radius?

\Answer{This would be indeed good to know. Unfortunately this is not easy to determine for this N-body simulation. The spiral arms in this simulation are transient. And we do not have enough simulation snapshots to determine the patter speed of these transient spiral arms. We hope however, that this is not a huge issue, given that we successfully analyse data drawn from anywhere between $R=1~\text{kpc}$ and $R=13~\text{kpc}$ and survey volumes of different sizes and at different positions. If the co-rotation would affect the modelling strongly, we would have noticed.} \hl{Elena, is there anything else, we can say about that?}

\paragraph{Aspect 5: Thick disk:} Another simplification of the N-body model is that there is no thick disk, while in
the MW, there is one which is about comparable mass to the thin disk. How will the
presence of this other components change the study of the orbits, and derivation of
the potential? \hl{How to answer to this?} 

\Answer{On "Aspect 5: Thick disk": Yes, we don't have a thick 
disk, but because by construction of the model the disk is very cold at 
2 scale -lengths our model has the maximal response of the disk to 
perturbations, and this is one of the conditions in which we wanted to 
test RoadMapping.} \hl{Not sure what Elena means.}

\paragraph{Aspect 6: No spiral arms outside of the solar radius.} Another issue is that the model considered has spiral arms only within the solar
radius, and is axisymmetric in the outer parts. Not surprisingly, there are big
departures from the true DF inside the solar radii, and less outside, as shown in
Figure 3. This will not be the case for a realistic MW model. \hl{The referee has a point that Figure 3 looks as if there were no spiral arms outside of the solar radius.}

\Answer{On "Aspect 6: No spiral arms outside of the solar radius": In the 1D spatial histograms the impression that there were no spiral arms outside of the solar radius is strengthened by averaging over the other two spatial coordinates.}

\paragraph{Aspect 2 \& 3 (continued):} p.19 in the conclusion: the authors conclude that their model has "has stronger spiral arms than we expect in the MW", which is not true, given the dominance of
the dark matter

\paragraph{Aspect 7: Number of spiral arms} It should also be recalled that in the MW when the true stellar component is mapped
in the NIR, the number of spiral arms turns out to be 2 only, while it is more 4
arms when Halpha and young stars are considered. This means that gas and star
formation are participating to harmonics and branching, but the basic stellar
structure is 2-armed. This will modify the conclusion, since the importance of
spiral arms in the stellar structure will be higher. \hl{How to answer to this?}

\paragraph{Task 1: Callibrating the method using smooth data.} There is also some issue which is striking in the paper, is that the departures from
the model are very high, when looking at Fig 3, 6 or even 11. And the reader wonders
how much is due already to the N-body model without spiral arms. Therefore it might
be interesting to calibrate the method, using the snapshot T=0 (or close to 0), the
initial axisymmetric model without spiral arms, and apply RoadMapping to understand
the  influence of the basic model. This might not be a long task, only providing a
test, or a Figure, but certainly will help to understand the high discrepancy
between derived results and the model.

%====================
\begin{figure}[!htbp]
\centering
\includegraphics[width=0.7\columnwidth]{fig/MNdHHinit_4kpc8Spiral_a_test1_data_bestfit_residuals_3b.pdf}
\caption{\Answer{???}}
\end{figure}
%====================

%====================
\begin{figure}[!htbp]
\centering
\includegraphics[width=0.7\columnwidth]{fig/MNdHHinit_4kpc8Spiral_a_test1_data_bestfit_residuals_3c.pdf}
\caption{\Answer{???}}
\end{figure}
%====================


\paragraph{Comment regarding the software policy:}

Per the new AAS software policy, http://journals.aas.org/policy/software.html, the
authors should use AASTeX v6.1 for the revised manuscript to highlight the code they
used with the new {\textbackslash}software command, e.g.

{\textbackslash}software{GADGET-3 (Springel et al. 2005), galpy (Bovy 2008), emcee (Foreman-Mackey
et al. 2013)}

\section{List of TO DOs for Wilma}


\begin{enumerate}
\item Include where fitting:  "A local spiral arm
density wave, for example, can impose kinematic fluctuations on the order of 10 km/s (Siebert et al. 2012)." from Bland-Hawthorn \& Gerhard 2016)
\end{enumerate}


\end{document}