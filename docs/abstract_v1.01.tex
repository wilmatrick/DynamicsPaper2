We are going to apply RoadMapping - an axisymmetric action-based dynamical modelling approach using smooth distribution functions (DFs) for Galaxy disk stars to recover the gravitational potential (Trick, Bovy, \& Rix 2016, submitted) - to the positions and velocities of star particles from a simulation snap-shot of a MW-like disk galaxy (D'Onghia et al.) with pronounced non-axisymmetries like spiral arms. We expect to find that for small survey volumes, in which one spiral arm strongly dominates the observed dynamics, our smooth model fails to recover the true potential. For large enough survey volumes that encompass more than just a few spiral arms, we hope however that their effect on the dynamics averages out and we are able to constrain the true potential to within 10%.
Because the simulation has no information about stellar chemical abundances, we treat the whole disk as a single stellar population. It would be very encouraging if we found that in this case fitting a superposition of only two quasi-isothermal DFs already allows for enough freedom to find a smooth equilibrium model for the entire disk. The use of more components or more complex, physically-motivated DFs could only seem justified, if we also explicitely treat spiral arms in the DF. We would like to propose a very simple outlier model for spiral arms, that can be included in the modelling to guide the fit towards a realistic Galaxy equilibrium model. It would be great if we could also show how such an equilibrium model helps to identify substructure in the disk.
And last but not least, we would like to present a numerical method in the appendix, that helps to calculate the smooth density distribution of a given DF and potential to high numerical accuracy efficiently and with much less computational cost than in previous implementations of RoadMapping. In addition we ague, that fitting a Stäckel potential directly to the data instead of using a physically-motivated galaxy potential has some advantages: Firstly, this is computationally *much* faster and secondly, the introduced error in the Action calculation is of similar or smaller size than other non-convergent Action calculation methods (see Sanders \& Binney 2015). This makes RoadMapping finally fit to be applied to real data.