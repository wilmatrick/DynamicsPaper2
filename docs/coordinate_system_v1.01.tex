\section{The Simulation}

\subsection{General information}

\begin{itemize}
\item \emph{Snapshot name:} snap\_
\item \emph{Snapshot time:} \Wilma{[TO DO]}
\item \emph{Reference:} \citet{2013ApJ...766...34D} \Wilma{[TO DO: I'm not sure...]}
\end{itemize}


\subsection{Particle Types}

\begin{itemize}
\item \emph{Particle Type 2: Stars.} 100 Million particles. Mean mass: $\sim 370 M_\odot$. Distributed in thick disk and shells.
\item \emph{Particle Type 3: \Wilma{[TO DO]}.} 10 Million particles. Mean mass: $\sim 950 M_\odot$. Distributed everywhere.
\item \emph{Particle Type 4: Giant Molecular clouds.} 1000 particles. Mean mass: $9.5 \cdot 10^5 M_\odot$. Distributed in thin disk. \citep{2013ApJ...766...34D}
\item \emph{Particle Types 0 and 1:} Do not exist in this simulation snapshot.
\end{itemize}


\subsection{Coordinate System}

Calculate moment of intertia (in \% units of largest entry):
\begin{itemize}
\item Star particles type 2, origin $= (0,0,0)$ (used in analysis for simplificity):
\begin{equation*}
\bar{\bar{I}} = \begin{pmatrix}
50.1027 &   0.1950 & -0.0074 \\
0.1950  &   50.2274 &  -0.0044 \\
-0.0074 &  -0.0044 &   100
\end{pmatrix}
\end{equation*}
\item Star particles type 2, origin $= (-0.007,0.002,-0.001)$ (particle position with deepest potential):
\begin{equation*}
\bar{\bar{I}} = \begin{pmatrix}
50.1027 &  0.1950 & -0.0074\\
 0.195 & 50.2273 & -0.0044\\
-0.0074 & -0.0044 & 100
\end{pmatrix}
\end{equation*}
\item Particles type 2, 3 and 4, origin $= (-0.008, 0.004, -0.003)$ (particle position with deepest potential):
\begin{equation*}
\bar{\bar{I}} = \begin{pmatrix}
88.598 &  0.090 & 0.015\\
 0.090 & 89.225 & 0.467\\
 0.015 &  0.467 & 100
\end{pmatrix}
\end{equation*}
\item \Wilma{[TO DO: find out, where the origin is of the static Hernquist halo.]}
\end{itemize}

\subsection{Disk Scale Length}

I fitted an radially exponential profile to the azimuthally averaged surface density of the star particles (2). I found 
$$R_s = 2.4993$$
As the Milky Way's disk scale length is also approximately 2.5 kpc, I can use the galaxy as it is. Lengths are in units of kpc, velocities in km/s, masses in $10^{10} M_\odot$.