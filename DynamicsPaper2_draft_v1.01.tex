%% Document class
\documentclass[iop,revtex4,numberedappendix,appendixfloats]{emulateapj}

%% General packages
\usepackage{amsmath}
\usepackage{mathrsfs }
\usepackage{lscape}

%% Figure packages
\usepackage{grffile}
\usepackage{subfigure}

%% Referencing
\usepackage{hyperref}

%% Custom macros
\newcommand{\vect}[1]{\boldsymbol{#1}}
\newcommand*\diff{\mathop{}\!\mathrm{d}}
\newcommand*\Diff[1]{\mathop{}\!\mathrm{d^#1}}
\newcommand{\pdf}{\ensuremath{pdf}}
\newcommand{\pmodel}{\ensuremath{p_M}}
\newcommand{\MAP}{MAP}
\newcommand{\MAPs}{MAPs}
\newcommand{\RM}{{\sl RoadMapping}}
\makeatletter
\newcommand{\testlabel}[2]{%
 \protected@write \@auxout {}{\string \newlabel {#1}{{#2}{\thepage}{#2}{#1}{}} }%
 \hypertarget{#1}{#2}
}
\makeatother

\usepackage[usenames,dvipsnames]{xcolor}
\newcommand{\Wilma}[1]{\textcolor{Magenta}{#1}}
\newcommand{\HW}[1]{\textcolor{Green}{#1}}
\newcommand{\Jo}[1]{\textcolor{Blue}{#1}}

%% Abbreviations
\shorttitle{The Influence of Spiral Arms on Action-based Dynamical Modelling of the Milky Way Disk}
\shortauthors{Trick et al.}

\begin{document}

%-----------------------------------------------------------------------------------------------------------------------------------------------------------------------------
%TITLE
%-----------------------------------------------------------------------------------------------------------------------------------------------------------------------------
\title{The Influence of Spiral Arms on Action-based Dynamical Milky Way Disk Modelling\\}

%% Authors
\author{Wilma H. Trick\altaffilmark{1,2}, Jo Bovy\altaffilmark{3}, Elena D'Onghia\HW{???}, and Hans-Walter Rix\altaffilmark{1}}

%% Affiliations
\altaffiltext{1}{Max-Planck-Institut f\"ur Astronomie, K\"onigstuhl 17, D-69117 Heidelberg, Germany}
\altaffiltext{2}{Correspondence should be addressed to trick@mpia.de.}
\altaffiltext{3}{Department of Astronomy and Astrophysics, University of Toronto, 50 St. George Street, Toronto, ON, M5S 3H4, Canada}

%-----------------------------------------------------------------------------------------------------------------------------------------------------------------------------
%ABSTRACT
%-----------------------------------------------------------------------------------------------------------------------------------------------------------------------------

\begin{abstract}
\begin{itemize}
\item One sentence on what RoadMapping is.
\item Overall axisymmetric RoadMapping modelling works in the presence of non-axisymmetric spiral arms, as long as the volume is big enough.
\end{itemize}
\end{abstract}

%-----------------------------------------------------------------------------------------------------------------------------------------------------------------------------
%KEYWORDS
%-----------------------------------------------------------------------------------------------------------------------------------------------------------------------------
\keywords{Galaxy: disk --- Galaxy: fundamental parameters --- Galaxy: kinematics and dynamics --- Galaxy: structure --- \Wilma{[TO DO]}}

%-----------------------------------------------------------------------------------------------------------------------------------------------------------------------------
%INTRODUCTION
%-----------------------------------------------------------------------------------------------------------------------------------------------------------------------------
\section{Introduction}

\begin{itemize}
\item Explain what RoadMapping is, also Acronym
\item Summarize BR13
\item Summarize results of Paper 1, mention that non-axisymmetries were not considered there
\item Main question: Does axisymmetric RoadMapping modelling work in the presence of non-axiysmmetric spiral arms?
\item Consequences: Both potential and orbit DF are not axisymmetric, i.e., the fitted axisymmetric potential model and DF do per se not contain the truth.
\item How to approach this: Use simulation by D'Onghia et al. 2013 and apply RM to it
\item The potential model we use is chosen mostly for practical reasons and is not necessarily the optimal one for the simulation. Also, we use a single qDF as DF - because it is the simplest thing to do. Also independently of the non-axisymmetries the chosen models might deviate from the truth. Where we investigated deviations between model and truth in isolated test cases, here several assumptions break down simultaneously.
\item Explain actions very shortly. $\vect{J}=(J_R,J_\phi=L_z,J_z)$ quantify oscillation in the coordinate directions $(R,\phi,z)$. Are calculated from current phase-space position in a given potential $\Phi$.
\item Say that actions are conserved in an axisymmetric potential, but not in non-axisymmetric potentials. (Maybe the mean vertical action is conserved \Wilma{[TO DO: Reference]}.) It is therefore important to check, if our modelling works in a system where actions are not conserved.
\end{itemize}

%-----------------------------------------------------------------------------------------------------------------------------------------------------------------------------
%MODELLING
%-----------------------------------------------------------------------------------------------------------------------------------------------------------------------------
\section{\RM{} modelling}

\subsection{Likelihood}

The data that goes into the modelling are the 6D position and velocity coordinates $(\vect{x}_i,\vect{v}_i)$ of $N_*$ stars within the survey volume. For simplicity we use a purely spatial selection function $\text{sf}(\vect{x})$ of spherical shape,
\begin{equation*}
\text{sf}(\vect{x}) \equiv \begin{cases} 1 &\mbox{if } \left| \vect{x}-\vect{x}_0 \right| \leq r_\text{max} \\
0 & \mbox{otherwise} \end{cases},
\end{equation*}
whose maximum radius $r_\text{max}$ defines the boundary of the survey volume and which is centred on $\vect{x}_0 \equiv (R_0,\phi_0,z_0=0)$. Given a parametrized potential model $\Phi(R,z)$ with parameters $p_\Phi$, the $i$-th star is on an orbit characterized by the orbital actions 
\begin{equation*}
\vect{J}_i \equiv \vect{J}(\vect{x}_i,\vect{v}_i \mid p_\Phi).
\end{equation*}
The total number of stars being on orbit $\vect{J}_i$ is given by an orbit distribution function $\text{df}(\vect{J})$ with parameters $p_\text{df}$,
\begin{equation*}
\text{df}(\vect{J}_i \mid p_\text{df}) \equiv \text{df}(\vect{J}(\vect{x}_i,\vect{v}_i \mid p_\Phi) \mid p_\text{df}) \equiv \text{df}(\vect{x}_i,\vect{v}_i \mid p_\Phi,p_\text{df}),
\end{equation*} 
where the latter equivalence arises from the Jacobian determinant between the angle-action coordinates $(\vect{\theta},\vect{J})$ and cartesian phase-space coordinates $(\vect{x},\vect{v})$, which is $\left| \partial (\vect{x},\vect{v}) / \partial(\vect{\theta},\vect{J})\right|=1$ and therefore allows us to treat the $\text{df}$ equivalently as a distribution of current phase-space coordinates or a distribution of orbital actions only, with uniform distribution in the angles $\vect{\theta}$.

\begin{itemize}
\item \Wilma{Write down likelihood formula}
\item \Wilma{Introduce outlier model as new aspect}
\item \Wilma{Refer to Paper 1 for details how to evaluate it, but mention shortly that it is a combination of nested-grid and MCMC}
\item \Wilma{Mention and reference galpy.}
\end{itemize}

\subsection{Potential and DF model}

\begin{itemize}
\item Introduce potential model, explain that form of disk was mostly chosen to the closed form expression of $Phi$ which allows for fast calculation. Both MNHH, DEHH and KKS pot.
\item Mention action calculation and that we tested explicitely that fixing Delta=0.45 and using staeckel interpolation grid does not degrade the analysis
\item Write down DF formula, simplest DF possible. Others use much more complicated ones.
\end{itemize}



%-----------------------------------------------------------------------------------------------------------------------------------------------------------------------------
%SIMULATION
%-----------------------------------------------------------------------------------------------------------------------------------------------------------------------------
\section{Data from a galaxy simulation}

%====================
\begin{figure*}[!htbp]
\plotone{fig/plot_simulation_for_paper.pdf}
\caption{}
\label{fig:???}
\end{figure*}
%====================

\subsection{Description of the galaxy simulation}

\subsection{Survey volume and data}

\begin{itemize}
\item Mention that we do not consider any measurement errors
\end{itemize}

\subsection{Symmetrized potential model}

\subsection{Quantifying influence of spiral arm}

%-----------------------------------------------------------------------------------------------------------------------------------------------------------------------------
%RESULTS
%-----------------------------------------------------------------------------------------------------------------------------------------------------------------------------
\section{Results}

\subsection{A single application of \RM{}}

\subsubsection{Fiducial test}

\begin{itemize}
\item $r_{max}=4kpc$
\item $N_*=20,000$
\item MNHH potential
\end{itemize}

\subsubsection{Recovering the stellar distribution}

\begin{itemize}
\item Figure: (x,y) and (R,z) distribution of residuals of true and best fit stellar distribution. Mark spiral arms as circles with radius Rg.
\item Figure: 1D histograms in R,z,phi, comparison of  true, best fit and best fit in symmetrized potential
\item Figure: 1D histograms in velocity and different (R,z,phi) bins comparison of  true, best fit and best fit in symmetrized potential
\end{itemize}

\subsubsection{Recovering the potential}

%====================
\begin{figure*}[!htbp]
\plotone{fig/MNdHHdiffSph_4kpc8Spiral_b_test1_density_overview.pdf}
\caption{Comparison of the true density distribution $\rho_{\Phi,\text{T}}$ in the galaxy simulation snapshot (solid black line, averaged over $\phi$) with the axisymmetric density distribution $\rho_{\Phi,\text{R}}$ recovered with \RM{} (solid blue lines) from $N_*=20,000$ stars in the survey volume with $r_\text{max}=4~\text{kpc}$ (yellow line), as described in Section \Wilma{[TO DO]}. The first two panels show density profiles along $(R,z=0)$ and $(R=8~\text{kpc},z)$, together with the relative differences between true and recovered $\rho_{\Phi}$. The third panel displays equidensity contours of the matter distribution in the $(R,z)$ plane. Overplotted are also the symmetrized ''true´´ potential's $\rho_{\Phi,\text{S}}$ (dotted black line) (see Section \Wilma{[TO DO]}) and the $\rho_{\Phi,\text{M}}$ of the recovered mean model in Table \Wilma{[TO DO]} (dotted blue line). The last panel shows the relative difference between the symmetrized ''true´´ $\rho_{\Phi,\text{S}}$ and the recovered mean model $\rho_{\Phi,\text{M}}$. Over wide areas even outside of the survey volume the relative difference is less than $10\%$. At $R\gtrsim8~\text{kpc}$ and $z\sim0$ it becomes apparent that the chosen potential model cannot perfectly capture the structure of the disk. \Wilma{[TO DO: Make sure that this plot actually contains the final analysis and sym. model that I want to show.]} \Wilma{[TO DO: Maybe it would be more interesting to see a best fit MNd directly to the potential to see, how well the potential model can actually perform?]} \Wilma{[TO DO: Maybe use only stars in the cone that the survey volume probes???]}}
\label{fig:???}
\end{figure*}
%====================

%====================
\begin{figure*}[!htbp]
\plotone{fig/MNdHHdiffSph_4kpc8Spiral_b_test1_vcirc_surfdens_overview.pdf}
\caption{Comparison of the circular velocity curve, surface density profile within $|z|\leq1.1~\text{kpc}$ and disk-to-halo ratio of the surface density along $R$ for the true potential of the galaxy simulation snapshot (solid black line) and the axisymmetric model potential recovered with \RM{} (solid blue lines) (see Section \Wilma{[TO DO]}). Overplotted are also the profiles of the symmetrized ''true´´ potential (dotted black line) (see Section \Wilma{[TO DO]}) and the recovered mean model (dotted blue line) (see Table \Wilma{[TO DO]}). The circular velocity curve is recovered to less than $5\%$, especially at larger radii. For the surface density and disk-to-halo ratio \RM{} recovers the truth at radii $\lesssim 8~\text{kpc}$. The deviations at larger radii are connected to the discrepancies in the density in Figure \Wilma{[TO DO]}. \Wilma{[TO DO: When I have the force I can probably also calculate the true circular velocity curve!]}}
\label{fig:???}
\end{figure*}
%====================



\begin{itemize}
\item Figure: density overview plot
\item Figure: vcirc, surfdens overview plot
\item Figure: local potential overview plot, scatter plot of stars color coded according to deviation of true and best fit (maybe also symmetrized) potential. normalize potential such that at solar circle pot=0. Both in \% of true potential and number of sigma away.
\item Figure: forces overview plot, incl. local forces scatter plot
\item Discuss somehow that the model parameters are actually themselves not very good recovered. Maybe violin plot?
\end{itemize}

\subsubsection{Recovering the action distribution}

\begin{itemize}
\item Figure: residuals in action space, comparison of true/symmetrized vs. best fit actions (maybe also true vs. best fit in symmetrized potential), overplot Lz=vcirc*Rg of spiral arms
\end{itemize}

\subsection{Investigation of different aspects}

\subsubsection{Test suite}

\begin{itemize}
\item $r_{max}=1,2,3,4,5kpc$
\item $N_*=20,000$
\item MNHH potential + KKS potential
\item $R_{obs} =5 and 8 kpc$
\end{itemize}

\subsubsection{Survey volume and choice of potential model}

\begin{itemize}
\item Figure: x-axis: $r_{max}$, y-axis: one panel with mean stellar rms deviation in FR and one with Fz. With different potentials and $r_{max}$.
\end{itemize}

\subsubsection{Influence of spiral arms}

\begin{itemize}
\item Figure: x-axis: $\langle \kappa \rangle$, y-axis: one panel with mean stellar rms deviation in FR and one with Fz. Analyses with same potential but at different positions and sizes within the galaxy.
\item Figure: x-axis: $sigma_\kappa$, y-axis: same as above figure.
\end{itemize}

%-----------------------------------------------------------------------------------------------------------------------------------------------------------------------------
%SUMMARY AND CONCLUSION
%-----------------------------------------------------------------------------------------------------------------------------------------------------------------------------
\section{Summary and conclusion}



\end{document}